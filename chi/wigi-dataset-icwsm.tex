\def\year{2016}\relax
%File: formatting-instruction.tex
\documentclass[letterpaper]{article}
\usepackage{aaai16}
\usepackage{times}
\usepackage{helvet}
\usepackage{courier}
\usepackage{graphicx}
\usepackage{booktabs} % for pretty table rules

\frenchspacing
\setlength{\pdfpagewidth}{8.5in}
\setlength{\pdfpageheight}{11in}
\pdfinfo{
/Title (Insert Your Title Here)
/Author (Put All Your Authors Here, Separated by Commas)}
\setcounter{secnumdepth}{0}  
 \begin{document}
% The file aaai.sty is the style file for AAAI Press 
% proceedings, working notes, and technical reports.
%
\title{Wikidata Human Gender Index: An Open Dataset}

\author{Max Klein \\
Grouplens, Department of Computer Science \\
Minneapolis, Minnesota 55455 \\
isalix@gmail.com\\
}
    
\maketitle
\begin{abstract}
AAAI creates proceedings, working notes, and technical reports directly from electronic source furnished by the authors. To ensure that all papers in the publication have a uniform appearance, authors must adhere to the following instructions. 
\end{abstract}


\section{Introduction}
Problematize introduction and make obvious need.

Landscape of ways in which Wikipedia shows bias, and Landscape of biases.

Our first observation is from September 17th 2014, and latest is January 3rd 2016. Although our dataset is now updated weekly following the official data dumps, automation was not completed until June 28th 2015, and so there is a window missing from October 2014 to June 2015. 

Albeit with caveats and huge biases, if the data is to believed, affords us an unprecedented look at gender representation on a time and geographic scale not possible before.

Introduce the dataset, and it's potential applications.	
\section{Dataset Details}

\begin{figure}
\label{fig:screenshot}
\includegraphics[scale=0.6]{figures/website_screenshot.png} 
\caption{Screenshot of wigi.wmflabs.org displaying the gender by culture plot.}
\end{figure}

How to use the data.

Our dataset is available for free under the CC0 license online at  http://wigi.wmflabs.org/ . Here a visualization demonstration of our data lives, and direct downloads, from the http://wigi.wmflabs.org/snapshot_data/ folder. 	This folder contains the CSV-formatted, weekly captured "snapshots" of gender-related data from Wikidata. Each snapshot is represented in its own folder, along with are two special folders. The special folders are titled \texit{newest} and \textit{newest-changes}, which will always link to the most recent snapshot, and the changes between the two most recent snapshots (week to week), respectively.

Each week, just after the release of Wikidata's official weekly data dump, we download a copy of the dump and process it in the following way. Using the Wikidtata Toolkit \cite{kroetsch} the entire database is subset to only those items which have the property \textit{instance of} with value \textit{human} - or ``P31:Q5" in Wikidata terms. For each human item we find the values of \texit{gender}, \textit{date of birth}, \textit{date of death}, \textit{place of birth}, \textit{place of birth}, \textit{place of birth}, \textit{place of birth}, \textit{place of birth} and \textit{place of birth}. These correspond to Wikidata properties Q1, Q1,Q1,Q1,Q1,Q1,Q1, respectively. In every snapshot folder this intermediate step is saved as ``gender-index-data-\{snapshot data\}.csv", it records one Wikidata item per row, and it's columns are the value of each property. In order to be faithfully represent Wikidata, the value of each property is actually a list, since Wikidata allows there to potentially be many competing values for a property. We store the list, inside the comma-separated sheet, as | (pipe)-separated values. 
 
For each of the recorded properties we then re-index the dataset aggregated on that property, but disaggregated by gender. So for instance, the date of birth index has one row per year unique year found as a date of birth, and one column per gender represented in Wikidata. For example the in the January 3^{rd} 2016 snapshot, the first row the date of birth index reads that for the year -4203 BCE Wikidata records two men, and no other genders as having this date of birth. Inside every snapshot folder there exists a subfolder containing 11 "re-indexes" of the snapshot - one per recorded property.
 
 The third item in every snapshot folder is titled ``changes-since- \{previous snapshot date\}, and contains computed differences between this snapshot and the last. This feature allows pinpoint what happened in the past week. There is one ``change-since" file for each property file. The operation simply takes the union of the row-index, and the columns, and subtracts the earlier values from the later values, element-wise. Using the date of birth example again, the changes since file will show which genders were added - or subtracted - for each year, in the course of the week.
   
How is it updated.
How can they be used. Examples of how to process the CSV can be found in our github repository (link.)

Inside the directory, you will find:

gender-index-data.csv (contains all properties)
property_indexes
per-property.csv
changes-since-2015
per-property-changes-since-last-week.csv

Show demonstrations of graphs from the website. As a demonstration the website displays 4 visualizations of the data. 


There is little cleaning done. More on this later.

\section{Data Statistics}
Total humans in Wikidata increased from 5,869,606 to 6,999,542, and shows linear, unconstrained growth.
\begin{figure}
\label{fig:totalhumans}
\includegraphics[scale=0.6]{figures/totalhumans.png} 
\caption{Total number of humans found in Wikidata at each snapshot period.}
\end{figure}

We should also be curious to the data quality of the increasing humans. One way to think about this is about the how much data is accompanying these human entries. We looked at the properties citizenship, place of birth, and ethnic group which will help us best geographically place a human. Another mark of quality is whether a human in Wikidata has an entry in a Wikipedia - a ``sitelink" in Wikidata vocabulary. \ref{fig:accompanying} shows the rate of accompanying properties, at the earliest and latest snapshots from 2014 and 2016 respectively. The statistics show that data quality has been increasing uniformly over time. The number of humans with \textit{gender} data increased by over 1\%, closer to complete coverage. Likewise \textit{citizenship} data increased by 15\% , \textit{place of birth} by 6\%  , and \textit{ethnic group} almost doubled \ref{table:accompanying}. Curiously the rate of humans having sitelinks decreased slightly. 

A Wikidata human without a Wikipedia article is know as a ``structural item"; for instance a Monarch without a Wikipedia article, but is a needed link in a family tree. With the view that a structural item is an artefact from humans paying attention to Wikidata's structure, the decrease in sitelinked humans can also be seen as a rise in data quality.

\begin{table}
\caption{Change in rates of human-accompanying properties}
\begin{tabular}{lrrrrr}
\toprule
snapshot &  gender &  citizenship &  place of birth &  ethnic group &  at least 1 site link \\
\midrule
2014-09-17 &  95.29\% &       42.82\% &          24.01\% &         0.31\% &                99.62\% \\
2016-01-03 &  96.54\% &       58.22\% &          30.51\% &         0.56\% &                98.15\% \\
\bottomrule
\end{tabular}

\label{table:accompanying}
\end{table}

\begin{figure}
\label{fig:accompanying}
\includegraphics[scale=0.6]{figures/additionalprops.png} 
\caption{Trend of human-accompanying properties by snapshot.}
\end{figure}

Another important factor to note is that our dataset tries not to clean itself to fit any model. In fact the ``gender" property in Wikidata is actually labelled in English ``sex or gender" (no distinction), and not limited to any value. Over our time recording we found 36 values used for ``sex or gender", including ``male" and ``female", but extending to nonbinary genders ``transgender female", ``intersex", ``fa'afafine", ``transgender", ``Gender fluid",  ``genderqueer", ``kathoey", and ``queer". At times the other information is recorded such as ``gay", or ``homosexuality" and ``Alien" or "cheetah". And even what seem to be mistakes are left in such as  ``Solanum tuberosum", ``Messi", or ``sociologist".

Focusing on one of our motivations, monitoring the trend of gender representation, we inspect the rate at which women are recorded in Wikidata. \ref{fig:frb} shows the ratio of ``female" recorded humans versus all gendered biographies. Similarly to total biographies this measure is rising at a fairly linear rate of approximately 0.5\% per year. The final months on record show a slight decline which warrants further investigation. In fact being able to measure at a level where is precisely one of the points of having such a live-updating measure - to be able to detect trends as they happen, and perhaps relate them to community issues. 

\begin{figure}
\label{fig:frb}
\includegraphics[scale=0.6]{figures/frbwikidata.png} 
\caption{Trend of human-accompanying properties by snapshot.}
\end{figure}

Another way in which Wikimedians can use the data is to look at trends specific to Wikipedia languages. It is easy to use the data to compile a top-10 list of Wikipedias whose female ratio of humans increased the most, see \ref{table:top10}. This measure does not explain whether women's representation in those languages increased because because editors took longer to record women's gender in Wikidata and were catching up in the observed period, or that these languages became more women-focused over the snapshotting period.

\begin{table}
\caption{Top 10 Wikis by increase in female ratio of humans from October 2014 to January 2016}
\label{table:top10}
\begin{tabular}{p{2cm}p{2cm}}
\toprule
{Wiki} &     Increase in female ratio of humans  \\
\midrule
Lithuanian      & 5.31\% \\
Japanese     & 4.76\% \\
Estonian      & 4.58\% \\
Slovenian      & 2.19\% \\
Tagalog      & 1.63\% \\
Korean      & 1.38\% \\
Finnish      & 1.33\% \\
Wikimedia Commons & 1.20\% \\
Farsi      & 1.17\% \\
Hebrew      & 1.17\% \\
\bottomrule
\end{tabular}


\end{table}

\section{Validation}
We validated out data by comparing it against 3 exogenous measures. Wikidata date of birth frequency versus historical world population trends, Wikidata gender by country  versus external by-country gender-disparity indexes, and Wikidata occupation gender versus United States Bureau of Labour Statistics occupation gender.

\subsection{World Population}

\subsection{External Indices}
Calibrated start dates were each 1900 or 1910 - a good sign. Over time the the WHGI-county is become more correlated to major external gender-disparity indexes \ref{table:scores}. In context with the fact that data quality is rising, this can be taken to mean that as Wikidata becomes more complete it is modelling the real world more. 

 \begin{table}
\caption{WHGI-country correlation to external indices. Correlation is the Spearman $\rho$, and signficances are *p<0.05, **p<0.01}
\label{table:scores}
\begin{tabular}{lrrrr}
\toprule
snapshot &  GEI &  SIGI &  GGGI &  GDI  \\
\midrule
2014-09-17 &  0.417** &       0.338** &          0.310* &         0.278**  \\
2016-01-03 &  0.457** &       0.402** &          0.386** &         0.299**  \\
\bottomrule
\end{tabular}
\end{table}


\section{Potential Applications}
Open Knowledge Foundation founder Rufus Pollock once said ``The best thing to do with your data will be thought of by someone else.”

FRB/wikisize. Is it data completeness or feminist-focus?


Use for determining women's notability for historical events.

Linking with VIAF. 

Comparing language's inherent gender bias to shown.

Wikimedian communities can use this an introspective watch.

Useful in all the same ways that external indices like the UNDP are too.


\section{ Acknowledgments}
We are especially grateful to the Wikimedia Foundation for funding us through an Individual Engagement Grant.

\bibliographystyle{aaai}
\bibliography{sample}

\end{document}
