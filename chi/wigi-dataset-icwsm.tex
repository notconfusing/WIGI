\def\year{2016}\relax
%File: formatting-instruction.tex
\documentclass[letterpaper]{article}
\usepackage{aaai16}
\usepackage{times}
\usepackage{helvet}
\usepackage{courier}
\usepackage{graphicx}
\usepackage{booktabs} % for pretty table rules
\usepackage{hyperref}
\usepackage[T1]{fontenc}


\frenchspacing
\setlength{\pdfpagewidth}{8.5in}
\setlength{\pdfpageheight}{11in}
\pdfinfo{
/Title Wikidata Human Gender Indicators: An Open Dataset with Perspectives on Times, Space, and Occupation
/Author BLINDED}
\setcounter{secnumdepth}{0}
 \begin{document}
% The file aaai.sty is the style file for AAAI Press
% proceedings, working notes, and technical reports.
%
\title{Wikidata Human Gender Indicators: An Open Dataset with Perspectives on Time, Space, and Occupation}

\author{AUTHORS BLINDED}

\maketitle
\begin{abstract}
We introduce the ``Wikidata Human Gender Indicators'' (WHGI), an open source, open data, real time, biographic dataset that can provide insights into gender disparities across time, space, culture, occupation and language. Leveraging the scale of Wikidata, the database that feeds Wikipedia, the WHGI is useful for researchers wishing to incorporate gender data in time periods and places for which other gender equality indices have no data. We present validations of the WHGI against three exogenous datasets: the world's historical population, ``traditional'' gender-disparity indices (GDI, GEI, GGGI and SIGI), and occupational gender according to the US Bureau of Labor Statistics. Presented are demonstrations of the dataset's format through visualizations, which also describe the rise of Wikidata's quality over time. Finally, we provide potential applications of how historians or linguists might use the dataset.
\end{abstract}


\section{Introduction}

Gender inequality is a long-standing social problem which affects many aspects of society. Worldwide, cultural ideologies have created scenarios which make women more prone to health issues \cite{world_health_organization_women_2009}. Likewise, in education attitudes create a systemic gender bias in opportunity \cite{heward_gender_1999}. Also, famously, incomes for identical jobs are lower for women \cite{burstein_equal_????}.

Statistical gender indicators are critically important to understanding gender inequality, but their construction is difficult \cite{klasen_gender-related_2004}. Many indicators and indices have been proposed, such as The Gender Development Index from the United Nations Development and the Global Gender Gap Index from the World Economic Forum, but no academic consensus exists on which is superior \cite{hawken_cross-national_2012}. Owing to this varied landscape using a plurality of indicators is recommended for research \cite{jutting_measuring_2008}.

We introduce an open source, open data, real-time, intersectional dataset that can provide insights into gender disparities across time, space, culture, occupation and language. It is called the ``Wikidata Human Gender Indicators'' (WHGI), and measures the gender of the humans represented in Wikidata - originating from Wikipedia biography articles - with their associated demographic data.

Measuring and releasing data on Wikipedia's content gender gap attempt to solve two problems. The scope of Wikipedia, and the advent of its machine-readability through Wikidata, gives us an unprecedented look at gender at a scale never seen before. It provides potential for applications not possible until now. Secondly, we wish to shed light on Wikipedia's data under the philosophy of ``what gets measured, gets fixed.''

The thread of research on Wikipedia's gender biases originates in the finding that Wikipedia's editors are largely not women \cite{hill_wikipedia_2013}. This imbalance has been attributed to an internet skills gap \cite{hargittai_mind_2015} and its internal culture \cite{lam_wp:clubhouse?:_2011}.

More and more thorough investigations of the Wikipedia \textit{editor gender gap} have also been interrogating the character of its \textit{biography gender gap}. Early studies found that Wikipedia excluded notable women more than its counterparts \cite{reagle_gender_2011}. More recently \cite{wagner_its_2015} showed that while coverage of women in large Wikipedias is not worse than other references, but the language with which women are portrayed is different and focuses more on romance and family. Women also tend to be less central in the link graph of Wikipedia \cite{10.1371/journal.pone.0114825}. These linguistic and network findings were confirmed by \cite{graells-garrido_first_2015}, who also showed evidence of stereotyping in the metadata.

In popular mindshare, however, there persists a sentiment that denies that any of this is a problem \cite{eckert_retriggering_2013}. Luckily, experiments are showing that awareness of Wikipedia's gender issues is a strategy that can alleviate the problem \cite{hinnosaar_gender_2015}, for which more methods are always needed.

Wikidata, the database that feeds Wikipedia, offers new opportunities to analyze culture programmatically. Launched in 2012, Wikidata is designed to host structured data that is multilingual (so there is only one edition) and plural (can support many competing facts) \cite{vrandecic_wikidata:_2014}.  These features make Wikidata the perfect place for all Wikipedias to collaboratively store facts about the world. If an Italian Wikipedian stores information about the population of ancient Rome, that information is then available to every other Wikipedia with a short code snippet. Every language collaborating together has meant that Wikidata has become a massive free open knowledgebase in its right, containing over 40 million facts \cite{krotzsch_how_????}.

As a knowledgebase, Wikidata is slowly proving its worth for research. For instance, Wikidata has been used to find popular connections between nationalities and occupations \cite{goldfarb_quantifying_2015}. Alternatively, take the fact that all human and mouse genes have been imported into Wikidata \cite{mitraka_wikidata:_2015} for an internet-wide community effort to find links between genes, drugs and diseases \cite{burgstaller-muehlbacher_wikidata_2015}. All of these tasks would be difficult to do without Wikidata.

\subsection{Outline}

This paper begins by describing the format of the data and presents a variety of statistics to illustrate how to use the data. We investigate how the data quality has changed over time, which demonstrates the 9 semantic properties available, and the advantages of its weekly, ``snapshot'' feature. We produce a top-10 of Wikipedia's which increased in their female ratio of human biographies as an example problem.

To show that WHGI is not entirely Wikipedia navel-gazing, we present 3 validation measures utilizing ground truths from the US Census Bureau, Bureau for Labor Statistics, and United Nations Development Program. We show that the WHGI does, in fact, relate to the real world. This means that, albeit imperfectly, the WHGI can be a proxy for real world numerical data in times and places for which no previous data exists.

Finally, the WHGI is meant for re-use, and we present some potential applications and scenarios demonstrating how it could be used in historical and linguistic research.

\section{Dataset Details}

Our dataset is available for free under the CC-BY-SA license online at  \url{WEBSITE BLINDED}. Here a visualization demonstration of our data lives along with portal organizing direct data downloads. 	This portal contains the CSV-formatted, weekly captured \textit{snapshots} of gender-related data from Wikidata. Each snapshot contains (1) an unaggregated one-row-per-human complete view of the data, (2) \textit{aggregated} ``property''-specific files, aggregated by tempo, spatial, and other properties, and (3) weekly changes between the property files. Our first snapshot is from September 17\textsuperscript{th} 2014, the latest presented here is January 3\textsuperscript{rd} 2016, but snapshotting is automated and tracks the official Wikidata data dumps.

\subsection{Processing}
Each week, just after the release of Wikidata's official weekly data dump, we download a copy of the dump and process it in the following way. Using the Wikidtata Toolkit \footnote{\url{https://www.mediawiki.org/wiki/Wikidata_Toolkit}} the entire database is subsetted to only those items which have the property \textit{instance of} with value \textit{human} - or ``P31:Q5'' in Wikidata terms.

For each human item we find the values of \textit{gender}, \textit{date of birth}, \textit{date of death}, \textit{place of birth}, \textit{citizenship}, \textit{ethnic group}, \textit{field of work}, and \textit{occupation}. These correspond to Wikidata properties P21, P569, P570, P19, P27, P172, P101, and P106 respectively.  In every snapshot folder this complete list of humans is saved as ``gender-index-data-\{snapshot date\}.csv'', it records one Wikidata item per row, and it's columns are the value of each property.

Because Wikidata is multilingual, its values are stored as identifiers - or ``Q-IDs'' in Wikidata terms - which can be translated into every language for which there is Wikipedia language edition. To maintain fidelity we keep this standard, so, for example, Aung San Suu Kyi represented in Wikidata in English looks like Figure \ref{fig:aung} and in our dataset would be a row like \begin{small} Q36740,1945,,Q6581072|,,Q836|,Q37995|,,Q82955|Q36180|Q1476215|
\end{small}. As a design decision we do not translate these Wikidata Q-IDs, to maintain language neutrality. We do however include functions to translate these Q-IDs into English (or any other language), which would render the above row as: \\
\begin{small} Aung San Suu Kyi,1945,,female|,,Myanmar|,Yangon|,,politician|writer|human rights activist|
\end{small}

\begin{figure}
\includegraphics[scale=0.15]{figures/aung_explainer.png}
\caption{Example Wikidata Human Item of Aung San Suu Kyi}
\label{fig:aung}
\end{figure}

For each of the recorded properties, we then aggregated the dataset on that property but disaggregated by gender. Take for example the date of birth file, it has one row per unique-year found as a date of birth, and one column per gender represented in Wikidata. See a sample excerpt from the January 3\textsuperscript{rd} 2016 snapshot in Table \ref{table:dob}. Additionally, there is an aggregate on the Wikipedia languages in which a human is represented - \textit{sitelinks}, and a geographic aggregation of citizenship, place of birth and ethnic group into properties called \textit{culture}, and \textit{worldmap}. Inside every snapshot directory, there exists a subdirectory titled ``property-indexes'' containing these 11 aggregates of the snapshot - one per property dimension.

\begin{table}
\caption{Excerpt of the 2016-01-03 date of birth-aggregate file.}
\begin{tabular} {p{0.8cm}p{0.8cm}p{0.8cm}p{0.8cm}p{0.8cm}p{0.8cm}p{0.8cm}}
\toprule
date of birth & no gender & trans-gender female & gender-queer & ka-thoey & female & male \\
\midrule
1980 & 839 & 2 & 1 & & 5,092 & 14,137   \\
1981 & 849 & 1 &  & 1 &5,042 & 14,461 \\
1982 & 861 & 2 &  & &5,132 & 14,372  \\
1983 & 864 & 3 &  & &5,078 & 14,520  \\
1984 & 830 & 3 & 1 & &5,372 & 14,558   \\
1985 & 777 & 4 &  & &5,400 & 14,664  \\
\bottomrule
\end{tabular}
\label{table:dob}
\end{table}

The third item in every snapshot folder is titled ``changes-since-\textit{\{previous snapshot date\}}'' and contains computed differences between this snapshot and the last. This feature allows inspection of the movements of the past week. There is one changes-since file for each property file. Using the date of birth example again, the changes-since file will show which genders were added - or subtracted - for each year, in the course of the week.

Examples of how to process the CSVs and generate the following results using both \textit{python-pandas} and \textit{R} can be found in our GitHub repository \footnote{\url{WEBSITE BLINDED}}.

\subsection{Online Visualization}

\begin{figure}
\includegraphics[scale=0.2]{figures/website_screenshot.png}
\caption{Screenshot of WEBSITE BLINDED displaying the gender by culture plot.}
\label{fig:screenshot}
\end{figure}

As a demonstration, our website displays 4 interactive visualizations of the dataset. Figure \ref{fig:screenshot} shows the \textit{gender-by-country} visualization, which shows the female ratio of humans by combined place of birth and citizenship, during the past week. Figure \ref{fig:screenshot2} shows the \textit{gender-by-language} visualization which compares the female ratio of biography by Wikipedia language, for the entirety of the dataset - ``all time''. The visualizations are interactive and allow the user to investigate specific parts of the plots. Each visualization is available both for all time and the previous week. We also produce top-10 information of each plot for readability. The remaining 2 unmentioned visualizations are \textit{gender-by-culture}, where culture is a further aggregation of countries and \textit{gender-by-date-of-birth-and-death}.

\begin{figure}
\label{fig:screenshot2}
\includegraphics[scale=0.25]{figures/screenshot2.png}
\caption{Screenshot of WEBSITE BLINDED displaying the gender by language plot.}
\end{figure}

Finally, there are two special directories in our portal which we offer for convenience,  titled \textit{newest} and \textit{newest-changes}, which will always link to the most recent snapshot, and the changes between the two most recent snapshots (week to week), respectively. This makes it easy to always access the latest data.

\subsection{Technical Details}
To faithfully represent Wikidata, the value of each property is a list, since Wikidata potentially allows multiple values for a property. This is because either two sources disagree on a property, or like in the case of Aung San Suu Kyi, she has many occupations, see Figure \ref{fig:aung}. We store the list inside a comma-separated sheet as | ``pipe''-separated values.

Of course, these multiple values introduce a design problem in aggregating on a list of properties. Our method is to aggregate on the list, rather than on the individual items within the list. This means in the case of Aung San Suu Kyi, that her occupation is stored as politician, writer, and human rights activist, and is aggregated with all the other humans who have those three occupations too. Since the dataset is open, interested researchers can use our raw data and aggregate it in any way they want.

Also note, for fidelity there is virtually no data-cleaning done, as the point of our project is to display information as faithfully as possible. Our dataset is meant to be used to uncover potential biases in Wikidata and the world at large, and we feel that any cleaning process would introduce further biases. An instructive illustration of this case is that the ``gender'' property in Wikidata is actually labelled in English  as ``sex or gender'' (no distinction), and not limited to any value. Over our time snapshotting we found 36 values used for ``sex or gender'', including ``male'' and ``female'', but extending to nonbinary genders ``transgender female'', ``intersex'', ``fa'afafine'', ``transgender'', ``Gender fluid'',  ``genderqueer'', ``kathoey'', and ``queer''. At times the other categories of information is recorded here - perhaps erroneously - such as ``gay'', or ``homosexuality''. And even what seem to be mistakes are left in, such as one-offs of ``Solanum tuberosum'', ``Messi'', or ``sociologist''. Cleaning this data would be a disservice, we feel, to communicating how - and how well - Wikidata is used.

\section{Dataset Statistics}
We now turn to look at simple statistics of our dataset, specifically with regard to how it has changed over time. Our first snapshot was on September 17\textsuperscript{th} 2014, and the latest analysed here is January 3\textsuperscript{rd} 2016. Since automation of snapshotting was not completed until June 28\textsuperscript{th} 2015, there is unfortunately a window missing from October 2014 to June 2015.

\subsection{Data Quality}

\begin{figure}
\includegraphics[scale=0.6]{figures/totalhumans.png}
\caption{Total number of humans found in Wikidata at each snapshot period.}
\label{fig:totalhumans}
\end{figure}

\begin{table}
\caption{Change in rates of property coverage for humans}
\begin{tabular}{lrr}
\toprule
{} &  2014-09-17 &  2016-01-03 \\
\midrule
gender               &       95.3\% &       96.5\% \\
date of birth        &       57.6\% &       71.7\% \\
date of death        &       28.6\% &       36.1\% \\
citizenship          &       42.8\% &       58.2\% \\
place of birth       &       24.0\% &       30.5\% \\
ethnic group         &        0.3\% &        0.6\% \\
field of work        &        n/a &        0.3\% \\
occupation           &        n/a &       58.7\% \\
at least 1 site link &       99.6\% &       98.1\% \\
\bottomrule
\end{tabular}
\label{table:accompanying}
\end{table}

Let us first query the size of the dataset. Total humans in Wikidata increased from 5,869,606 to 6,999,542, and shows linear, unconstrained growth (see Figure \ref{fig:totalhumans}).
We should also be curious to the data quality of the increasing humans. One way to think about this coverage of properties on these human items. We investigated the coverage completeness for all  properties we recorded. Another mark of quality is whether a human in Wikidata has an entry in a Wikipedia - a ``sitelink'' in Wikidata vocabulary. Figure \ref{fig:accompanying} and Table \ref{table:accompanying} show the trend in coverage of properties from the earliest and latest snapshots. The statistics show that data quality has been increasing almost uniformly over time. The number of humans with \textit{gender} data increased by over 1\%, closer to complete coverage. In the time domain \textit{date of birth} and \textit{date of death} coverage increased by 14\% and 7\%. Likewise, \textit{citizenship} data increased by 15\%, \textit{place of birth} by 6\%, and \textit{ethnic group} almost doubled. \textit{Field of work}, and \textit{occupation} data was not included in our dataset until late, so their growth, while increasing is not precisely comparable.

Curiously the rate of humans having sitelinks decreased slightly, but this has an important interpretation. A Wikidata human without a Wikipedia article is known as a ``structural item''; for instance a member of royalty without a Wikipedia article but is a needed to make a family tree complete. With the view that a structural item is an artefact from humans paying attention to Wikidata's structure, the decrease in sitelinked humans can also be seen as an increase in data quality.

Probably there are also biography articles in Wikipedias that are not semantically described as humans in Wikidata, however this set is difficult to enumerate since knowing that a Wikipedia article is about a human is difficult to do programmatically.


\begin{figure}
\includegraphics[scale=0.5]{figures/additionalprops.png}
\caption{Trend of property coverage by snapshot (time).}
\label{fig:accompanying}
\end{figure}

\subsection{Simple Metrics}
As further motivation for how the dataset can be used, we present some simple metrics which would be of interest to Wikipedian communities. Focusing on one of our motivations, monitoring the trend of gender representation, we inspect the rate at which women are recorded in Wikidata. Figure \ref{fig:frb} shows the ratio of ``female'' recorded humans versus all gendered biographies. Similarly to total biographies this measure is rising at a fairly linear rate of approximately 0.5\% per year. The final months on record show a slight decline which warrants further investigation. In fact being able to measure at this temporal level is precisely one of the points of having such a live-updating measure - to be able to detect trends as they happen, and perhaps relate them to community issues.

\begin{figure}
\includegraphics[scale=0.6]{figures/frbwikidata.png}
\caption{Female ratio of humans in Wikidata by snapshot (time).}
\label{fig:frb}
\end{figure}

Another way in which Wikipedians can use the data is to look at trends specific to Wikipedia languages. It is easy to use the dataset to compile a top-10 list of Wikipedias whose female ratio of humans increased the most, see Table \ref{table:top10}. This measure does not explain whether women's representation in those languages increased, because editors took longer to record women's gender in Wikidata and were catching up in the observed period, or that these languages became more women-focused over the snapshotting period. Such forensics is possible with this dataset though.

\begin{table}
\caption{Top 10 Wikis by increase in female ratio of humans from October 2014 to January 2016. English Wikipedia included as a baseline.}
\label{table:top10}
\begin{tabular}{p{2cm}p{2cm}}
\toprule
{Wiki} &     Increase in female ratio of humans  \\
\midrule
Lithuanian      & 5.31\% \\
Japanese     & 4.76\% \\
Estonian      & 4.58\% \\
Slovenian      & 2.19\% \\
Tagalog      & 1.63\% \\
Korean      & 1.38\% \\
Finnish      & 1.33\% \\
Wikipedia Commons & 1.20\% \\
Farsi      & 1.17\% \\
Hebrew      & 1.17\% \\
English      & 0.48\% \\
\bottomrule
\end{tabular}
\end{table}

\section{Validation}
In order to gain an idea about how well WHGI reflects the real world we validated our data by comparing it against 3 exogenous datasets. We correlated the WHGI by date of birth versus historical world population trends; WHGI by country versus exogenous gender-disparity indices; and WHGI by occupation versus United States Bureau of Labor Statistics occupation by gender.

\subsection{World Population} Our most simple validation measure involved comparing the world's population by year to the number of humans in WHGI by year of birth. We conduct this validation even though, the number of people alive and the number of Wikipedia-notable people born are different measures. However, if we operate under the assumptions that (a) the proportion of the world population which is Wikipedia-notable is constant over time and (b) that the birth rate is a fixed proportion of the population, then theoretically their curves should share approximately the same shape.

We performed a standard Pearson correlation between the number of people in Wikidata born in a particular year, and the estimated historical world population by the US Census
Bureau\footnote{\url{https://commons.Wikipedia.org/wiki/File:Population_curve.svg}}.
We also conducted this correlation for our earliest and latest snapshots. The results in Table \ref{table:worldpop} show a high and significant correlation between real world estimates and Wikidata, at about 0.85. We do see a very minor decrease in correlation over snapshots of 0.007. Overall though the population of Wikidata over time seems very aligned with the World's population over time.

\begin{table}
\caption{Correlation of number of WHGI by date of birth and world population. Significances are $ ^{**}p\leq 0.01$.}
\label{table:worldpop}
\begin{tabular}{lrrrr}
\toprule
snapshot &  Pearson correlation \\
\midrule
2014-09-17 & 0.852**  \\
2016-01-03 & 0.845**  \\
\bottomrule
\end{tabular}
\end{table}

\subsection{Exogenous Gender-Disparities Indices}
\begin{table}
\caption{WHGI-country correlation to external indices. Correlation is the Spearman $\rho$, and significances are $ ^*p\leq 0.05 $, $ ^{**}p\leq 0.01$.}
\label{table:scores}
\begin{tabular}{lrrrr}
\toprule
snapshot &  GEI &  SIGI &  GGGI &  GDI  \\
\midrule
2014-09-17 &  0.417** &       0.338** &          0.310* &         0.278**  \\
2016-01-03 &  0.457** &       0.402** &          0.386** &         0.299**  \\
\bottomrule
\end{tabular}
\end{table}

WHGI is inspired, in part, by the rich landscape of gender disparity indices. This type of index ranks countries by a measure of gender equality. If we aggregate WHGI by place of birth and citizenship, and look at the female ratio of humans, we too have a sort of country-by-gender equality measure\footnote{Despite having the same by-country unit of analysis with this aggregation WHGI is not an ``index'' like those we compare it to, since an index weights and combines many indicators \cite{rossi_handbook_1980}. }.  We correlated the country rankings of this WHGI aggregation with 4 popular exogenous indices to see how well Wikidata reflects real world gender disparities.

The 4 exogenous indices we used were: The United Nations' Gender Development Index (\textbf{GDI})  \footnote{\url{http://hdr.undp.org/en/content/gender-development-index-gdi}},  Social Watch's Gender Equity Index (\textbf{GEI}) \footnote{\url{http://www.socialwatch.org/node/14366}},  the Global Gender Gap Index (\textbf{GGGI}) \footnote{\url{http://reports.weforum.org/global-gender-gap-report-2014/}}, and the Social Institutions and Gender Index (\textbf{SIGI}) \footnote{\url{http://www.genderindex.org/ranking}}.
These exogenous indices measure indicators like life expectancy, years of schooling, earned income, representation in parliament and senior executive positions, and social norms.

Additionally, we conducted a calibration step, to find the date of birth threshold which maximized our correlations. In each case the maximizing threshold was found to be between 1900 and 1910. We interpreted the found thresholds as a good sign firstly because the exogenous indices are measures of recent history too, and secondly because it shows a robustness in the way that WHGI relates to exogenous indices.

Table \ref{table:scores} shows the correlations with each index, all of which were significant and ranged from 0.278 to 0.457. Affirmingly, when looking at this information through a snapshotting lens, the correlation with every index is increasing over time. On the low end the GDI correlation grew by 7.6\%, and on the upper end, the SIGI correlation jumped 24.5\% in the about-a-year time frame. That is, the gender disparities found in WHGI by country are increasingly looking more like the real world gender disparities.


\subsection{Occupation Gender}
The notion of what a human's job or occupation is, we saw in Table \ref{table:accompanying}, well recorded in Wikidata. To answer the question of how accurate Wikidata's gender representation by occupation is, we compared it to data from the United States Bureau of Labor Statistics (BLS) \footnote{\url{http://www.bls.gov/cps/aa2012/cpsaat11.htm}}. We borrow this ground truth technique from \cite{kay_unequal_2015} who used it to evaluate the gender representation of Google image search results.

Approximately 60\% of our sample have occupation data, and together over 4,000 occupations are represented. The BLS has 332 occupation categories which are at a higher level ontologically than what is recorded in Wikidata. Whereas Wikidata might record that someone is a pastry chef, the BLS only has a category for cooks. In order to match the datasets we used Wikidata's internal ontology hierarchy, to generalize the occupation terms. A ``subclass of'' property exists in Wikidata, that relates items to their more general concept - and this is true for occupations as well. Wikidata describes that pastry chef is a subclass of chef, and that chef is a sublcass of cook.

Our method was to raise the generality of Wikidata occupations until there were less than 500 occupations to ease the matching task. Two authors then matched occupations manually for accuracy and confirmation. We resolved disagreements until the sets were matched. However not all occupations could be matched due to the specificity of the BLS, rendering coverage of Wikidata occupations 57\% complete. The largest occupations in Wikidata were sportsperson and politician, and neither of them had matches in the BLS. In the reverse, there were many BLS occupations for which Wikidata did not have any matching occupations, such as ``lodgings manager'', which outlines a limitation of this validation, that being a lodgings manager does not inherently make you notable for inclusion in Wikipedia. It must be acknowledged too that the BLS data describes the United States whereas WHGI has a worldwide scope, which may explain why we found no significant correlation in the \textit{size} of the matching occupations between the two sets.

Finally we correlated the rankings of the list of most gendered occupations according to WHGI to that of the BLS. We did this for early and late snapshots, but because occupation was not a property that we initially recorded, our first snapshot which included occupation was August 9\textsuperscript{th} 2015.  Table \ref{table:bls} shows the spearman rank correlation found was a significant 0.410, and since then the correlation has increased noticeably to 0.473. These are moderate correlations which we claim support a link that Wikidata reflects the real world.

\begin{table}
\caption{Rank correlation of gender ratios by occupation between WHGI and US
Bureau of Labor Statistics. Signficances are $ ^{**}p\leq 0.01$.}
\begin{tabular}{lrrrr}
\toprule
snapshot &  Spearman Rank Correlation \\
\midrule
2015-08-09 & 0.410**  \\
2016-01-03 & 0.473**  \\
\bottomrule
\end{tabular}
\label{table:bls}
\end{table}

\section{Representative Limitations}
The WHGI measures two phenomena that we do not disentangle: human development and Wikipedia content development. On the one hand, the validation of the dataset tells us how well our measurements capture the various dimensions of gender equality and human development. On the other hand, we are also inspecting how Wikipedia's biography articles and notability policy are based on the real world.

To some degree, WHGI represents the real world. In each of our validation measures, we found high or moderate correlations. Certainly WHGI is not random or isolated but captures real world dynamics with some distortion. We notice both that WHGI validations are rising over time, and that data quality is rising over time. This can be taken to mean that as Wikidata becomes more complete, it is modelling the real world more. There is some justification in using WHGI as a proxy for real-world phenomena, but that proxy is limited by the worldview of Wikipedia editors and constrained by its notability policies.

Wikipedia's notability policies require humans to be in positions power which are systemically biased against women (AUTHOR BLINDED), how then can the rise in women's biographical representation be explained? This research cannot answer that question, but we suggest several possible reasons. At least three factors that affect encyclopedic inclusion are: (1) the rate at which women receive positions of power in the real world, (2) the level of gender bias in Wikipedias' notability policies, and (3) the level of efforts to write about women in Wikipedia. Anyone or all three of these factors can contribute to the trend we see in Figure \ref{fig:frb}.

\section{Potential Applications}
Though we believe in what Open Knowledge Foundation founder Rufus Pollock once said ``the best thing to do with your data will be thought of by someone else,'' we still present a few ideas to stoke creativity. Since the data arises from Wikipedia, its introspective uses are many, but we also propose uses that are completely unrelated to the Wiki-universe.

Wikipedian communities can use WHGI to measure their editorial and content focus. With the temporal nature of the snapshotting, applications could be built to detect spikes in creation or deletion of types of humans. This could be useful to measure the effectiveness of editathons, and other planned editing. However such a tool could also alert to the presence of unplanned activity, good or bad, which effects the macro-level gender of Wikipedia and Wikidata.

Divorced from Wikipedia entirely, a historian could use the data to determine the gender-disparity levels of a specific place and time. Typically to quantify the gender climate one would rely on the indices like those mentioned in the Exogenous Indices section. However these indices, are limited to discussing recent history. With large caveats about accuracy, our validation showed that our data is in touch with the real world. With this dataset we can quantify a type of gender-disparity of medieval France, ancient Greece, or Ming dynasty China. WHGI is useful in all the same ways that exogenous indices are used, only with a larger timespan. That is certainly a novel approach not possible before Wikidata.

Additionally, another new avenue this dataset opens is in the gender-disparity of a language. A linguist could use WHGI aggregated by language to quantify the gendered-ness of a language. Furthermore with the date of birth and death information, that linguist could see how differently languages focus on gender over time. Potentially this could lend evidence to another theory of language that comes from their native methods.

Generally, WHGI is, in essence, a biographic database. The data can not only provide insights on gender-related disparity, but also other disparities such as culture disparities, citizen disparities and ethnic group disparities, etc.

\section{Conclusion and Future Work}
We made the Wikidata Human Gender Indicators (WHGI), a biographic database for researchers wishing to incorporate gender data along dimensions of time, space, and occupation. Based off of Wikidata and Wikipedia, it can most obviously be used by that community to monitor demographic trends and biases in their content. We also validated the indicators with measures of the real world, such as population, country-based gender disparities, and occupations. These validations showed that the WHGI is significantly correlated to real world demographics and gender disparities. We also showed that data quality of Wikidata has been increasing. Data quality and correlations increasing together is particularly encouraging the support for using WHGI as a tool. WHGI is freely available for download, we have outlined some of the potential ways in which it could be used and hope that many more are thought of by others.

We hope to continue running the open source project in service of the Wikipedia and research communities seeking to statistically describe gender disparities. The ways in which we expand WHGI we hope will be directed by user's feedback.


\section{ Acknowledgments}
We are especially grateful to the Wikimedia Foundation for funding us through an Individual Engagement Grant. [GRANT BLINDED].

\bibliographystyle{aaai}
\bibliography{wigi-dataset-icwsm}

\end{document}
