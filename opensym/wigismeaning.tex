% This is "sig-alternate.tex" V2.0 May 2012
% This file should be compiled with V2.5 of "sig-alternate.cls" May 2012
%
% This example file demonstrates the use of the 'sig-alternate.cls'
% V2.5 LaTeX2e document class file. It is for those submitting
% articles to ACM Conference Proceedings WHO DO NOT WISH TO
% STRICTLY ADHERE TO THE SIGS (PUBS-BOARD-ENDORSED) STYLE.
% The 'sig-alternate.cls' file will produce a similar-looking,
% albeit, 'tighter' paper resulting in, invariably, fewer pages.
%
% ----------------------------------------------------------------------------------------------------------------
% This .tex file (and associated .cls V2.5) produces:
%       1) The Permission Statement
%       2) The Conference (location) Info information
%       3) The Copyright Line with ACM data
%       4) NO page numbers
%
% as against the acm_proc_article-sp.cls file which
% DOES NOT produce 1) thru' 3) above.
%
% Using 'sig-alternate.cls' you have control, however, from within
% the source .tex file, over both the CopyrightYear
% (defaulted to 200X) and the ACM Copyright Data
% (defaulted to X-XXXXX-XX-X/XX/XX).
% e.g.
% \CopyrightYear{2007} will cause 2007 to appear in the copyright line.
% \crdata{0-12345-67-8/90/12} will cause 0-12345-67-8/90/12 to appear in the copyright line.
%
% ---------------------------------------------------------------------------------------------------------------
% This .tex source is an example which *does* use
% the .bib file (from which the .bbl file % is produced).
% REMEMBER HOWEVER: After having produced the .bbl file,
% and prior to final submission, you *NEED* to 'insert'
% your .bbl file into your source .tex file so as to provide
% ONE 'self-contained' source file.
%
% ================= IF YOU HAVE QUESTIONS =======================
% Questions regarding the SIGS styles, SIGS policies and
% procedures, Conferences etc. should be sent to
% Adrienne Griscti (griscti@acm.org)
%
% Technical questions _only_ to
% Gerald Murray (murray@hq.acm.org)
% ===============================================================
%
% For tracking purposes - this is V2.0 - May 2012

\documentclass{sig-alternate}
\usepackage{url}

\begin{document}
%
% --- Author Metadata here ---
\conferenceinfo{OpenSym}{'15 San Francisco, Calif. USA}
%\CopyrightYear{2007} % Allows default copyright year (20XX) to be over-ridden - IF NEED BE.
%\crdata{0-12345-67-8/90/01}  % Allows default copyright data (0-89791-88-6/97/05) to be over-ridden - IF NEED BE.
% --- End of Author Metadata ---

\title{Wikipedia in the World of Global Gender Inequality Indices: What The Biography Gender Gap Is Measuring}
%
% You need the command \numberofauthors to handle the 'placement
% and alignment' of the authors beneath the title.
%
% For aesthetic reasons, we recommend 'three authors at a time'
% i.e. three 'name/affiliation blocks' be placed beneath the title.
%
% NOTE: You are NOT restricted in how many 'rows' of
% "name/affiliations" may appear. We just ask that you restrict
% the number of 'columns' to three.
%
% Because of the available 'opening page real-estate'
% we ask you to refrain from putting more than six authors
% (two rows with three columns) beneath the article title.
% More than six makes the first-page appear very cluttered indeed.
%
% Use the \alignauthor commands to handle the names
% and affiliations for an 'aesthetic maximum' of six authors.
% Add names, affiliations, addresses for
% the seventh etc. author(s) as the argument for the
% \additionalauthors command.
% These 'additional authors' will be output/set for you
% without further effort on your part as the last section in
% the body of your article BEFORE References or any Appendices.

\numberofauthors{1} %  in this sample file, there are a *total*
% of EIGHT authors. SIX appear on the 'first-page' (for formatting
% reasons) and the remaining two appear in the \additionalauthors section.
%
\author{
% You can go ahead and credit any number of authors here,
% e.g. one 'row of three' or two rows (consisting of one row of three
% and a second row of one, two or three).
%
% The command \alignauthor (no curly braces needed) should
% precede each author name, affiliation/snail-mail address and
% e-mail address. Additionally, tag each line of
% affiliation/address with \affaddr, and tag the
% e-mail address with \email.
%
% 1st. author
\alignauthor
Max Klein
       \email{max@notconfusing.com}
       }

\maketitle
\begin{abstract}
The Wikipedia editor gender gap is important but difficult to measure, but its biographical gender gap can readily produce measurements.
We correlate a Wikipedia-derived gender inequality indicator (WIGI), with four widespread gender inequality indices in use today (GDI, GEI, GGGI, and SIGI). Analysing their methodologies and correlations to Wikipedia, we find evidence that Wikipedia's notability policy is a contributing factor to gender inequality in biographical coverage.
\end{abstract}

% A category with the (minimum) three required fields
\category{J.4}{SOCIAL AND BEHAVIORAL SCIENCES}{Sociology}
%A category including the fourth, optional field follows...
\category{K.6.2}{MANAGEMENT OF COMPUTING AND INFORMATION SYSTEMS}{Installation Management}[Performance and usage measurement]

\terms{Human Factors}

\keywords{data mining, Wikidata, Wikipedia, gender gap, demographics}

\section{Introduction}
Encylopedias have long contained gender bias, both in their \textit{editorship} and in their \textit{biographical coverage}. \cite{thomas:respect} \cite{reagle}. Like its historical counterparts, Wikipedia has been estimated to have a female authorship of around only 13\% - 16\% \cite{ghosh} \cite{hill}, a point that has seen media coverage \cite{eckert}. Even though there may be no connection between the two biases, a skew of the same order of magnitude has been found by studies of biographical articles in Wikipedia \cite{lam}, \cite{eom}, \cite{reagle}, \cite{klein:blog}.

One attempt at quantifying Wikipedia's biographical gender gap has been called \textit{Wikipedia Gender Inequality Indicator ("WIGI")}. WIGI is a set of measures of gender in Wikipedia by date of birth and death, place of birth and citizenship, and inclusion in different Wikipedia language editions \cite{klein:wigi}. In particular,  the measure of biographical gender by place of birth or citizenship can be seen as an inequality index ranking countries by gender representation and therefore can be compared to other inequality indices that are used in global reports. These comparisons could shed light on the constitution of Wikipedia.

\section{Indices}

While there are many different gender gap indices with their own methodologies, there is a consensus that the complexity of the issue means that no single index is the best \cite{mills} \cite{hawken} \cite{beneria}. We use four ``general" indices that are in widespread use. 

\begin{itemize}
\item The United Nations Development Programme's Gender-related Development Index (\textbf{GDI}) was introduced only in 1995. It is a gender-focused extensions of the Human Development Index. GDI's primary focus lies in gender gaps in life expectancy, education, and incomes.
\item The Gender Equity Index (\textbf{GEI}) was introduced by Social Watch in 2005.  The GEI was developed to measure many situations that are unfavourable to women, it ranks countries on three dimensions: education, economic participation and empowerment.
\item The Global Gender Gap Index (\textbf{GGGI}), developed by the World Economic Forum in 2006. The GGGI is intended to allow comparative comparison of gender gap across different countries and years, it focuses on four areas:  economic participation and opportunity, educational attainment, political empowerment and health statistics.
\item The Social Institutions and Gender Index (\textbf{SIGI}) is of the OECD Development Centre from 2007. A composite indicator of gender equality that solely focuses on social institutions (norms, values and attitudes), SIGI uses the five dimensions of discriminatory family code, restricted physical integrity, son bias, restricted resources and assets, and restricted civil liberties.
\end{itemize}

In relation to these indices we will compare the country-focused part of \textbf{WIGI}. WIGI is derived entirely from Wikidata. Wikidata is a database that stores and feeds semantic facts to Wikipedia. Its October 14 2014 dataset included a total of 1,061,634 Wikidata items about humans with a \textit{date of birth} and a \textit{citizenship} or a \textit{place of birth} property that is a country or a city linked to a country. Place of birth and citizenship we recognize may not be the same. In cases of contradiction we simply select one at random, as we are looking at large aggregates we don't believe this will have a significant effect. For every country the \textit{ratio of female and nonbinary gendered biographies} to total biographies born-in of citizen-of is the score\cite{klein:wigi}.  

\section{Research Question}
Given the literature on this subject we ask:\\
RQ1: Of the other general gender indices which index is Wikipedia least and most closely related to?

\section{Method}

In each index higher scores represent higher equality, however in order to normalise comparisons we focus only on the positional rank of the countries, not the precise score. See table \ref{table:topten} for an example comparative ranking of WIGI's top ten with GGGI. To understand how close two indices are we use the Spearman rank correlation coefficient. In order to avoid missing data problems we use the intersection of the countries covered by both indices. 

We next compute a calibration step on the year of birth in biographies to find the start decade to subset WIGI which maximises the correlation between WIGI and a general index. Note that the date ranges for the general indices are not in question; only which biographies to include in the WIGI index. Also note that the end date for biography inclusion is held constant at the present date. We are looking at what time until present the biography data of Wikidata makes a by-country index that most matches a general gender index.

For each general index we compute the maximum Spearman rank correlation and the start decade that achieves it. Then of all the general indices we find the one whose maximum achievable correlation is highest and lowest. Of the highest and lowest general indices we compare and contrast the measurement methodologies. 
Full data\footnote{\url{https://github.com/notconfusing/WIGI/blob/master/helpers/foreign_indexes/WIGI_comparison.csv}}
 and code\footnote{\url{http://nbviewer.ipython.org/github/notconfusing/WIGI/blob/master/World
 Economic Forum Comparison.ipynb}} available online.
\begin{table}
\caption{WIGI's top-ten rank being compared to GGGI.}
\label{table:topten}
\begin{tabular}{|l|c|c|}
\hline
Country &
GGGI Rank &
WIGI Rank \\\hline
Sweden &
4 &
1 \\\hline
South Korea &
117 &
2 \\\hline
Philippines &
9 &
3 \\\hline
Bahrain &
124 &
4 \\\hline
Mauritius &
106 &
5 \\\hline
People's Republic of China &
87 &
6 \\\hline
Australia &
24 &
7 \\\hline
Japan &
104 &
8 \\\hline
Nicaragua &
6 &
9 \\\hline
Swaziland &
92 &
10 \\\hline
\end{tabular}
\end{table}

\section{Results}

We produce a comparison table of indices,  their correlation, the correlation significance, and the maximizing start decade, ordered by correlation, in table \ref{table:scores}. Each general index shows some statistically significant moderate correlation with WIGI. This is a sanity check that the female and nonbinary ratio of Wikidata humans associated with a country is, at minimum, a legitimate addition to the landscape of gender inequality indices.

Also we see that the calibrated start decades are all similar. At about 1910 for each, we see a clear signal about the fact that what date range general indices talk about. Figure \ref{fig:evo} shows an example of how the correlation with GGGI spikes at this date.


\begin{figure}
\includegraphics[scale=0.5]{spearman_evolution_gggi.png}
\caption{Evolution of rank correlation by start decade for GGGI}
\label{fig:evo}
\end{figure}

\begin{table}
\centering
\caption{WIGI's correlation to general indices and calibrated start date.}
\label{table:scores}
\begin{tabular}{|l|p{2cm}|l|p{2cm}|}
\multicolumn{4}{}{ WIGI compared to Alternative indices}\\\hline
Index &
Spearman Correlation &
Significance &
Calibrated Start Decade\\\hline
GEI &
0.417 &
p{\textless}0.001 &
1910\\\hline
SIGI &
0.338 &
p{\textless}0.001 &
1910\\\hline
GGGI &
0.310 &
p=0.03 &
1890\\\hline
GDI &
0.278 &
p{\textless}0.001 &
1910\\\hline
\end{tabular}
\end{table}




\section{Discussion}
First we find it very affirming that each general index most highly correlates with WIGI around 1910. Intuitively this makes sense in light of the fact that general indices are more measure of modern history, they seek to measure the present day. The present day humans are those born about 1910 or later. Wikidata has information dating back past 1000 BCE, but this information is not useful in describing the world that the general indices do. 

What is the interpretation that our nation-WIGI is most highly correlated to GEI, and least with GDI? What do GEI and GDI measure that show what WIGI is measuring? We dig further into the methodologies of theses indices. Social Watch's GEI explains itself that:

\begin{quote}
    ``In Education, GEI looks at the gender gap in enrolment at all levels and in literacy; economic participation computes the gaps in income and employment and empowerment measures the gaps in highly qualified jobs, parliament and senior executive positions."\footnote{\url{http://www.socialwatch.org/node/14366}}
\end{quote}

And the UN's GDI reports itself as:

\begin{quote}
   ``The new GDI measures gender gap in human development achievements in three basic dimensions of human development: health, measured by female and male life expectancy at birth; education, measured by female and male expected years of schooling for children and female and male mean years of schooling for adults ages 25 and older; and command over economic resources, measured by female and male estimated earned income."\footnote{\url{http://hdr.undp.org/en/content/gender-development-index-gdi}}
 \end{quote}

We find that both indices share some overlapping features in education and income, but diverge on the features of empowerment and health. 
In their similarities of education and income we find related measurements of school enrolment, years of schooling and earned income.
The disparity that emerges is that GEI additionally measures empowerment by positions of power whereas GDI additionally measures life expectancy. Since GEI is the is the most similar to WIGI and GDI is the least similar, this suggests that the WIGI is more highly correlated to women's positions of power by country than to life expectancy by country. 

At first glance, this finding is commensurate Wikipedia's notability policies. Notability in Wikipedia, although it varies by language, essentially defers to inclusion or absence in the journalistic and scholarly record. That means that humans in positions of power, as GEI imports, would would have articles in Wikipedias in greater proportion, because more powerful positions are more covered in media and scholarship. Thinking about GDI's uniqueness in life expecetancy, one does not obviously see a strong reason that those with greater life expectancy are more covered in Wikipedia. Higher longevity may mean that the chances of doing something notable increases as the time to earn media attention increases, but this is not as direct a link. 

Finally we would conclude that WIGI and Wikipedia's representation bear more semblance to the gendered differences in highly qualified jobs, parliament and senior executive positions than longevity. That link might be clear with some feminist reasoning, but the data also supports the notion.

\section{Limitations and Future Work}
The analysis of manually inspecting the least and most correlated indices is naive and manual. A more sophisticated system might be able to understand each index as a set of measures, and then compute what (linear) combination of indices best correlate with WIGI. In this way we would have a more nuanced analysis of which measures WIGI is uncovering.

%
% The following two commands are all you need in the
% initial runs of your .tex file to
% produce the bibliography for the citations in your paper.
\bibliographystyle{abbrv}
\bibliography{sigproc}  % sigproc.bib is the name of the Bibliography in this case
% You must have a proper ".bib" file
%  and remember to run:
% latex bibtex latex latex
% to resolve all references
%
% ACM needs 'a single self-contained file'!
%
\end{document}  % This is where a 'short' article might terminate
